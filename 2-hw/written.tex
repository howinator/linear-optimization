\documentclass[12pt]{article}
 \usepackage[margin=1in]{geometry}
\usepackage{amsmath,amsthm,amssymb,amsfonts,algorithm,algpseudocode,algorithmicx,xfrac}

\newcommand{\N}{\mathbb{N}}
\newcommand{\Z}{\mathbb{Z}}
\newcommand{\matr}[1]{\mathbf{#1}}

\newenvironment{problem}[2][Problem]{\begin{trivlist}
\item[\hskip \labelsep {\bfseries #1}\hskip \labelsep {\bfseries #2.}]}{\end{trivlist}}
\newenvironment{subproblem}[1]{\textbf{(#1)}}{}

\theoremstyle{definition}
\newtheorem{definition}{Definition}[section]

\newtheorem{theorem}{Theorem}[section]
\newtheorem{corollary}{Corollary}[theorem]
\newtheorem{lemma}[theorem]{Lemma}
%If you want to title your bold things something different just make another thing exactly like this but replace "problem" with the name of the thing you want, like theorem or lemma or whatever

\begin{document}

%\renewcommand{\qedsymbol}{\filledbox}
%Good resources for looking up how to do stuff:
%Binary operators: http://www.access2science.com/latex/Binary.html
%General help: http://en.wikibooks.org/wiki/LaTeX/Mathematics
%Or just google stuff

\title{Linear Optimization - Homework 2}
\author{Howie Benefiel \(phb337\)}
\maketitle

\begin{problem}{1}
\begin{subproblem}{1$^{st}$ Bullet}
$ $ \newline
If $\matr{A}\matr{d}=0$, that means $\matr{d}$ is in the nullspace of $\matr{A}$.
This means that the vector $\matr{d}$ points in a direction which minimizes all variables.
Heuristically, this observation can be used to argue that only moving in direction which are in the nullspace of $A$ lead to feasible solutions.

Formally, following the argument on pg. 83,
if $\matr{d}$ is a feasible direction at $\matr{x}$, that means $\exists  \theta > 0$ s.t. $\matr{x} + \theta \matr{d} \in$ the feasible polyhedron, $P$.
Since we're only looking for feasible solutions, $\matr{A}(\matr{x} + \theta \matr{d})=\matr{b}$.
Also, since $x$ is feasible, $\matr{A}\matr{x} = \matr{b}$.
Since, $\theta > 0$ and subtracting $b$ from both sides, it is apparent $Ad=0$.

Now for the second part of the statement, we're moving in a feasible direction in the polyhedron,
so $\matr{x}+\theta \matr{d} \geq 0$.
That means for a given index, $i$, $x_i+\theta d_i \geq 0$.
Since $\theta > 0$, $d_i \geq 0$ for any index where $x_i=0$. \\
\end{subproblem}

\begin{subproblem}{2$^{nd}$ Bullet}
$ $ \\
We should follow the same analysis as above.

If we can move in the d direction then $\matr{A}(x + \theta \matr{d}) = \matr{b}$.
Since $\matr{A}\matr{x}=\matr{b}$, those terms cancel leaving $\theta \matr{A} \matr{d} = 0$.
We divide out $\theta$ leaving $\matr{A}\matr{d}=0$.
This proves the first statement of the set.

For, $\matr{D}\matr{d} \leq 0$, we again start with the definition of moving in a feasible direction.
This gives us $\matr{D}(\matr{x}+\theta\matr{d}) \leq \matr{f}$.
From the problem statement, $\matr{D}\matr{x} \leq \matr{f}$, so those terms cancel.
That leaves us with $\theta\matr{D}\matr{d} \leq 0$.
We divide out $\theta$, leaving us with the second statement defining the set, $\matr{D}\matr{d} \leq 0$.
\end{subproblem}

\end{problem}


\begin{problem}{2}
\begin{subproblem}{1$^{st}$ Bullet}
The extreme points will be on the axes of the polyhedron.
That means the extreme points will be $\{(0,0,\sfrac{1}{3}), (0, \sfrac{1}{2}, 0), (1, 0, 0), (0, 0, 0)\}$.
\end{subproblem}

\begin{subproblem}{2$^{nd}$ Bullet}
Assuming cost can not be negative,
\begin{align*}
\matr{x} = \begin{pmatrix}0 \\ 0 \\ \sfrac{1}{3} \end{pmatrix} \implies \matr{c}=\begin{pmatrix}c_1 \\ c_2 \\ 0 \end{pmatrix}
\forall c_1, c_2 \\
\matr{x} = \begin{pmatrix}0 \\ \sfrac{1}{2} \\ 0 \end{pmatrix} \implies \matr{c}=\begin{pmatrix}c_1 \\ 0 \\ c_3 \end{pmatrix}
\forall c_1, c_3 \\
\matr{x} = \begin{pmatrix}1 \\ 0 \\ 0 \end{pmatrix} \implies \matr{c}=\begin{pmatrix}0 \\ c_2 \\ c_3 \end{pmatrix}
\forall c_2, c_3 \\
\matr{x} = \begin{pmatrix}0 \\ 0 \\ 0 \end{pmatrix} \implies \matr{c}=\begin{pmatrix}c_1 \\ c_2 \\ c_3 \end{pmatrix}
\forall c_2, c_3
\end{align*}
\end{subproblem}
\end{problem}

\begin{problem}{3.12}

\begin{subproblem}{a}
In standard form, our problem is described as

\begin{align*}
\text{minimize \hspace{0.5in}} -2x_1-x_2 \\
\text{st \hspace{0.5in}} x_1-x_2+x_3 = 2 \\
x_1 + x_2 + x_4 = 6 \\
x_1, x_2, x_3, x_4 \geq 0
\end{align*}
BFS of $(x_1, x_2) = (0, 0)$ yields a cost of 0 and $x_3 = 2$ and $x_4 = 6$.
\end{subproblem}

\begin{subproblem}{b}
\begin{equation}
\begin{array}{c}
\\
  \\
x_3 \\
x_4
\end{array}
\begin{bmatrix}
\begin{array}{c|cccc}
    & x_1 & x_2 & x_3 & x_4 \\ \hline
  0 & -2 & -1 & 0 & 0 \\ \hline
  2 & 1^* & -1 & 1 & 0  \\
  6 & 1 & 1 & 0 & 1 \\
\end{array}
\end{bmatrix}
\end{equation}
$ $ \\
\begin{equation}
\begin{array}{c}
\\
  \\
x_1 \\
x_4
\end{array}
\begin{bmatrix}
\begin{array}{c|cccc}
    & x_1 & x_2 & x_3 & x_4 \\ \hline
  4 & 0 & -3 & 2 & 0 \\ \hline
  2 & 1 & -1 & 1 & 0  \\
  4 & 0 & 2^* & -1 & 1 \\
\end{array}
\end{bmatrix}
\end{equation}
$ $ \\
\begin{equation}
\begin{array}{c}
\\
  \\
x_1 \\
x_2
\end{array}
\begin{bmatrix}
\begin{array}{c|cccc}
    & x_1 & x_2 & x_3 & x_4 \\ \hline
  10 & 0 & 0 & 0.5 & 1.5 \\ \hline
  6 & 1 & 0 & .5 & .5  \\
  2 & 0 & 1 & -0.5 & 0.5 \\
\end{array}
\end{bmatrix}
\end{equation}
That means we have an optimal solution at $x_1=6, x_2=2$.

\end{subproblem}

\begin{subproblem}{c}
Graph goes here.
\end{subproblem}


\end{problem}

\begin{problem}{4}
\begin{equation}
\begin{array}{c}
\\
  \\
y_1 \\
y_2 \\
y_3
\end{array}
\begin{bmatrix}
\begin{array}{c|cccccccc}
    & x_1 & x_2 & x_3 & x_4 & x_5 & y_1 & y_2 & y_3 \\ \hline
  -5 & -1 & -1 & -3 & -1 & -2 & 0 & 0 & 0 \\ \hline
  2 & 1 & 3 & 0 & 4 & 1^* & 1 & 0 & 0  \\
  2 & 1 & 2 & 0 & -3 & 1 & 0 & 1 & 0  \\
  1 & -1 & -4 & 3 & 0 & 0 & 0 & 0 & 1  \\
\end{array}
\end{bmatrix}
\end{equation}
$ $ \\

\begin{equation}
\begin{array}{c}
\\
  \\
x_5 \\
y_2 \\
y_3
\end{array}
\begin{bmatrix}
\begin{array}{c|cccccccc}
    & x_1 & x_2 & x_3 & x_4 & x_5 & y_1 & y_2 & y_3 \\ \hline
  -1 & 1 & 5 & -3 & 7 & 0 & 2 & 0 & 0 \\ \hline
  2 & 1 & 3 & 0 & 4 & 1 & 1 & 0 & 0  \\
  0 & 0 & -1 & 0 & -7 & 0 & -1 & 1 & 0  \\
  1 & -1 & -4 & 3^* & 0 & 0 & 0 & 0 & 1  \\
\end{array}
\end{bmatrix}
\end{equation}
$ $ \\

\begin{equation}
\begin{array}{c}
\\
  \\
x_5 \\
y_2 \\
x_3
\end{array}
\begin{bmatrix}
\begin{array}{c|cccccccc}
    & x_1 & x_2 & x_3 & x_4 & x_5 & y_1 & y_2 & y_3 \\ \hline
  0 & 0 & 1 & 0 & 7 & 0 & 2 & 0 & 1 \\ \hline
  2 & 1 & 3 & 0 & 4 & 1 & 1 & 0 & 0  \\
  0 & 0 & -1^* & 0 & -7 & 0 & -1 & 1 & 0  \\
  \sfrac{1}{3} & -\sfrac{1}{3} & -\sfrac{4}{3} & 1 & 0 & 0 & 0 & 0 & \sfrac{1}{3}  \\
\end{array}
\end{bmatrix}
\end{equation}
$ $ \\

We still have $y_2$ in the basis which we must drive out.
We apply a change of basis to drive it out.

\begin{equation}
\begin{array}{c}
\\
  \\
x_5 \\
x_2 \\
x_3
\end{array}
\begin{bmatrix}
\begin{array}{c|cccccccc}
    & x_1 & x_2 & x_3 & x_4 & x_5 & y_1 & y_2 & y_3 \\ \hline
  0 & 0 & 0 & 0 & 0 & 0 & 1 & 1 & 1 \\ \hline
  2 & 1 & 0 & 0 & -17 & 1 & -2 & 0 & 0  \\
  0 & 0 & 1 & 0 & 7 & 0 & 1 & 1 & 0  \\
  \sfrac{1}{3} & -\sfrac{1}{3} & 0 & 1 & \sfrac{28}{3} & 0 & 0 & 0 & \sfrac{1}{3}  \\
\end{array}
\end{bmatrix}
\end{equation}
$ $ \\

This means our Phase I BFS is $(x_5, x_2, x_3) = (2, 0, \sfrac{1}{3})$.
We now re-introduce the original objective function.

\begin{equation}
\begin{array}{c}
\\
  \\
x_2 \\
x_3 \\
x_5
\end{array}
\begin{bmatrix}
\begin{array}{c|ccccc}
    & x_1 & x_2 & x_3 & x_4 & x_5 \\ \hline
  6 & 1 & 0 & 0 & -82 & 0 \\ \hline
  0 & 0 & 1 & 0 & 7 & 0  \\
  2 & -\sfrac{1}{3} & 0 & 1 & \sfrac{28}{3}^* & 0  \\
  \sfrac{1}{3} & 1 & 0 & 0 & -17 & 1  \\
\end{array}
\end{bmatrix}
\end{equation}
$ $ \\

\begin{equation}
\begin{array}{c}
\\
  \\
x_2 \\
x_4 \\
x_5
\end{array}
\begin{bmatrix} % This needs to be finished (top row is done)
\begin{array}{c|ccccc}
    & x_1 & x_2 & x_3 & x_4 & x_5 \\ \hline
  \sfrac{660}{28} & -\sfrac{218}{28} & 0 & \sfrac{246}{28} & 0 & 0 \\ \hline
  -\sfrac{3}{2} & \sfrac{1}{4}^* & 1 & -\sfrac{3}{4} & 0 & 0  \\
  \sfrac{3}{14} & -\sfrac{1}{28} & 0 & \sfrac{3}{28} & 1 & 0  \\
  \sfrac{167}{42} & \sfrac{11}{28} & 0 & \sfrac{51}{28} & 0 & 1  \\
\end{array}
\end{bmatrix}
\end{equation}
$ $ \\

\begin{equation}
\begin{array}{c}
\\
  \\
x_1 \\
x_4 \\
x_5
\end{array}
\begin{bmatrix} % This needs to be finished (top row is done)
\begin{array}{c|ccccc}
    & x_1 & x_2 & x_3 & x_4 & x_5 \\ \hline
  -\sfrac{162}{7} & 0 & \sfrac{218}{7} & -\sfrac{102}{7} & 0 & 0 \\ \hline
  -6 & 1 & 4 & -3 & 0 & 0  \\
  0 & 0 & 0 & 0 & 1 & 0  \\
  \sfrac{19}{3} & 0 & -\sfrac{11}{7} & 3^* & 0 & 1  \\
\end{array}
\end{bmatrix}
\end{equation}
$ $ \\

\begin{equation}
\begin{array}{c}
\\
  \\
x_1 \\
x_4 \\
x_3
\end{array}
\begin{bmatrix} % This needs to be finished (top row is done)
\begin{array}{c|ccccc}
    & x_1 & x_2 & x_3 & x_4 & x_5 \\ \hline
  \sfrac{160}{21} & 0 & \sfrac{778}{49} & 0 & 0 & \sfrac{34}{7} \\ \hline
  \sfrac{1}{3} & 1 & -\sfrac{4}{7} & 0 & 0 & 0  \\
  0 & 0 & 0 & 0 & 1 & 0  \\
  \sfrac{19}{9} & 0 & -\sfrac{11}{21} & 1 & 0 & \sfrac{1}{3}  \\
\end{array}
\end{bmatrix}
\end{equation}
$ $ \\

\end{problem}


\end{document}
